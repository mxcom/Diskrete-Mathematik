\subsection{Rekursionsgleichungen}

\textbf{1. Gleichung umstellen:}

\begin{itemize}
\item Rekursionsgleichung in einen homogenen (links) und einen inhomogenen Teil  (rechts) teilen
\item \textbf{Homogener Teil:} Überall wo ein $a$ vorhanden ist
\item \textbf{Inhomogener Teil:} Rest wo kein $a$ vorhanden ist
\end{itemize}
$$
homogener \ Teil = inhomogener \ Teil
$$

\textbf{2. Allgemeine Lösung des homogenen Teils berechnen:}

\begin{itemize}
\item Das Charakteristisches Polynom aufstellen indem man alle $a_n$ mit $\lambda ^n$ ersetzt
\end{itemize}
$$
p(\lambda ) = x_m \lambda ^{n_m} + x_{m-1} \lambda ^{n_{m-1}} + \dots x_0 + \lambda ^{n_0}
$$

\begin{itemize}
\item Das Charakteristisches Polynom lösen, durch finden der Nullstellen 
\begin{itemize}
\item Bei Polynom zweiten Gerades ($\lambda ^2$) pq-Formel verwenden: $-\frac{-p}{2} \pm \sqrt{(\frac{p}{2})^2 - q}$
\end{itemize}
\end{itemize}
$$
x_m \lambda ^{n_m} + x_{m-1} \lambda ^{n_{m-1}} + \dots x_0 + \lambda ^{n_0} = 0
$$

\begin{itemize}
\item Allgemeine Lösung für homogenen Teil setzt sich aus den Nullstellen $\lambda _1, \lambda _2, \dots , \lambda _k$ zusammen
\begin{itemize}
\item Falls die Nullstellen verschieden (und reelle Zahlen) sind ist die allgemeine Lösung des homogenen Teils: $$a_{h,n} = c_1 \lambda _1 ^n + c_{2} \lambda _2 ^n + \dots + c_k \lambda _k ^n$$
\item Falls die Nullstellen gleich (und relle Zahlen) sind ist die allgemeine Lösung des homogenen Teils: $$a_{h,n} = c_1 \lambda _1 ^n + c_{2} n \lambda _2 ^n + c_{3} n^2 \lambda _3 ^n + \dots + c_k n^{k-1} \lambda _k ^n$$
\end{itemize}
\end{itemize}

\textbf{3. Ansatz für Spezielle Lösung des inhomogenen Teils herleiten:}

\begin{itemize}
\item Mit Hile der Tabelle lässt sich ein Ansatz für die spezielle Lösung des inhomogenen Teils herleiten (vermutete spezielle Lösung)
\item Dieser wird mit $a_{s,n}$ bezeichnet
\end{itemize}

\begin{table}[h]
\renewcommand{\arraystretch}{1.5}
\begin{tabular}{l|l}
& Ansatzfunktion: $y_{s,n}$\\
\hline
$b \ (Konstante)$ & $B \ (Konstante)$\\
$n$ & $B_1 n + B_0$\\
$n^t, t \in \mathbb{N}$ & $B_t n^t + B_{n-1} n^{t-1} + \dots + B_1 n + B_0$\\
$r^n, r \in R$ & $Br^n$\\
$n^t r^n$ & $r^n (B_t n^t + B_{n-1} n^{t-1} + \dots + B_1 n + B_0)$
\end{tabular}
\end{table}

\textbf{4. Ansatz der Spezielle Lösung in den homogenen Teil einsetzen, um die spezielle Lösung zu erhalten:}

\begin{itemize}
\item Ansatz der Spezielle Lösung in den homogenen Teil einsetzen
\item Koeffizientenverlgeich zum bestimmen der speziellen Lösung
\end{itemize}\

\textbf{5. $a_n$ bestimmen}

\begin{itemize}
\item Kombiniere die allgemeine Lösung des homogenen Teils und die spezielle Lösung des inhomogenen Teils: $$a_n = a_{h, n} + a_{s, n}$$
\end{itemize}\

\textbf{6. Gegebenes $a_0$ und $a_1$ benuten}

\begin{itemize}
\item Gegebenen Anfangswerte in die allgemeine Lösung eingesetzt, um die spezifischen Werte der Konstanten zu bestimmen, die benötigt werden, um die endgültige Lösung zu finden.
\end{itemize}\
\\

\hrule

\textbf{Spezialfälle:}

\begin{itemize}
\item[1.] Falls in der allgemeinen Lösung des homogenen Teils, es gleichen Stellen wie in dem Ansatz der speziellen Lösung der inhomogenen Lösung gibt, dann wird ein $n$ im Ansatz der speziellen Lösung hinzugefügt
\item \textbf{Beispiel:} $$a_{h,n} = c_1 {\color{red}3^n} + c_2(-2)^n$$ \begin{center}
\textit{und}\end{center} $$a_{s,n} = b_1 {\color{red}3^n} + b_2 n + b_3$$ \begin{center} \textit{Vor $3^n$ wird deshalb ein $n$ hinzugefügt} \end{center} $$a_{s,n} = b_1 {\color{blue}n}{\color{red}3^n} + b_2 n + b_3$$
\item[2.] Falls es ein $1^n$ in der allgemeine Lösung des homogenen Teils gibt mit $n \geq 0$, dann wird ein n im speziellen Teil hinzugefügt.
\item \textbf{Beispiel:} $$a_{h,n} = c_1 {\color{red}3^n} + c_2(-2)^n$$ \begin{center}
\textit{und}\end{center} $$a_{s,n} = b_1 {\color{red}1^n} + b_2 n + b_3$$ \begin{center} \textit{Vor $3^n$ wird deshalb ein $n$ hinzugefügt} \end{center} $$a_{s,n} = b_1 {\color{blue}n}{\color{red}3^n} + b_2 n + b_3$$
\end{itemize}

\newpage

\textbf{Beispiel:} $a_{n+2} = 49 a_n + 48n - 98, \qquad n \geq 0 \qquad a_0 = 5, a_1 = 8$

\textbf{1. Gleichung umstellen}
$$
a_{n+2} - 49a_n = 48n - 98
$$

\textbf{2. Allgemeine Lösung des homogenen Teil berechnen}

\begin{itemize}
\item Ersetze alle $a_n$ mit $\lambda ^n$
\item $a_{n+{\color{blue}2}} \rightarrow \lambda ^{\color{blue}2}$ und $-49a_{n+{\color{blue}0}} \rightarrow -49 \lambda ^{{\color{blue}0}} = -49$
\end{itemize}
$$
p(\lambda) =  \lambda ^2 - 49
$$

\begin{itemize}
\item Nullstelle bestimmen ($\lambda _1$ und $\lambda _2$)
\end{itemize}
$$
\lambda ^2 - 49 = 0 \rightarrow \lambda _{1,2} = \pm 7
$$

\begin{itemize}
\item Allgemeine Lösung für homogenen Teil:
\end{itemize}
$$
\boxed{a_{h,n} = c_1 7^n + c_2 (-7)^n}
$$

\textbf{3. Ansatz für spezielle Lösung des inhomogenen Teil}

\begin{itemize}
\item Nach Tabelle Ansatz für spezielle Lösung aufstellen
\end{itemize}

$$
a_{s,n} = b_1 n + b_2
$$

\begin{table}[h]
\renewcommand{\arraystretch}{1.5}
\begin{tabular}{l|l}
& Ansatzfunktion: $y_{s,n}$\\
\hline
$b \ (Konstante)$ & $B \ (Konstante)$\\
$n$ & $B_1 n + B_0$\\
$n^t, t \in \mathbb{N}$ & $B_t n^t + B_{n-1} n^{t-1} + \dots + B_1 n + B_0$\\
$r^n, r \in R$ & $Br^n$\\
$n^t r^n$ & $r^n (B_t n^t + B_{n-1} n^{t-1} + \dots + B_1 n + B_0)$
\end{tabular}
\end{table}

\textbf{4. Ansatz für spezielle Lösung in homogenen Teil einfügen}

\begin{itemize}
\item homogener Teil: $a_{{\color{blue}n+2}} - 49a_{\color{red}n}$
\item Ansatz spezielle Lösung: $a_{s,n} = b_1 n + b_2$
\item inhomogener Teil: $48n - 98$
\end{itemize}
\begin{align*}
b_1({\color{blue}n+2}) + b_2 - 49(b_1{\color{red}n} + b_2) &= 48n - 98\\
b_1n + 2b_1 + b_2 - 49b_1 n - 49b_2 & = 48n - 98
\end{align*}

\begin{itemize}
\item Nun vergleicht man die Koeffizienten der speziellen Lösung $(n^1, n^0)$
\item inhomogener Teil: ${\color{Green}48}n {\color{orange}- 98}$
\end{itemize}
\begin{align*}
n^1&: b_1 -49b_1 = -48b_1 = {\color{Green}48} \rightarrow b_1 = -1\\~\\
n^0&: 2b_1 + b_2 - 49 b_2 = {\color{orange}-98} \rightarrow b_2 = 2
\end{align*}

\begin{itemize}
\item Nun setzt man die ermittelten Werte in den Ansatz der speziellen Lösung ein und erhält die spezielle Lösung
\end{itemize}
$$
\boxed{a_{s,n} =  -1n + 2}
$$

\textbf{$a_n$ bestimmen}
\begin{align*}
a_n &= a_{h,n} + a_{s,n}\\
a_n & = c_1 7^n + c_2 (-7)^n -1n + 2
\end{align*}

\textbf{$a_0$ und $a_1$ benutzen um $c_1$ und $c_2$ zu bestimmen}

\begin{itemize}
\item $a_0 = 5, a_1 = 8$
\end{itemize}

\begin{align*}
a_0: c_1 7^0 + c_2(-7)^0 - 1 \cdot 0 + 2 &= 5\\
c_1 - c_2 + 2 &= 5
\end{align*}

\begin{align*}
a_1: c_1 7^1 + c_2(-7)^1 - 1\cdot 1 + 2 &= 8\\
7c_1 - 7c_2 -1 + 2 &= 8
\end{align*}

\begin{itemize}
\item $a_0$ nach $c_1$ umstellen un in $a_1$ einsetzen um $c_2$ zu bestimmen
\item dann $c_2$ in $c_1$ einsetzen um $c_1$ zu erhalten
\item Werte von $c_1$ und $c_2$ in $a_n$ einsetzen, das Ergebnis ist die allgemeine Lösung
\end{itemize}

\begin{align*}
a_0: c_1 - c_2 + 2 &= 5\\
c_1 &= 3 + c_2
\end{align*}

\begin{align*}
a_1: 7(3 + c_2) - 7c_2 -1 + 2 &= 8\\
\dots
\end{align*}

$$
\boxed{a_n = c_1 7^n + c_2 (-7)^n -1n + 2}
$$

Hier $c_1$ und $c_2$ mit eigentlichen Werten ersetzen

\newpage

\subsection{Typische Klausurfragen zu Rekursionsgleichungen:}

\textbf{1. Typ der Rekursionsgleichung bestimmen}

\begin{itemize}
\item \textbf{Homogene lineare Rekursion:}
\begin{itemize}
\item Besteht nur aus $a$'s (kann auch ein anderer Buchstabe sein) aber es werden keine Koeffizienten vorgegeben
\item \textbf{Aufbau:} $$a_{n+k} = y_{k-1}a_{n+k-1} + y_{k-2}a_{n+k-2} + \dots + y_{1}a_{n+1} + y_{0}a_{n} \qquad n \geq 0$$
\item \textbf{Beispiel:} $$a_{n+2} = 3a_{n+1} - 2a_n \qquad n \geq 0$$
\end{itemize}
\item \textbf{Homogene lineare Rekursion mit konstanten Koeffizienten:}
\begin{itemize}
\item Besteht nur aus $a$'s (kann auch ein anderer Buchstabe sein)
\item \textbf{Aufbau:} $$a_{n+k} = y_{k-1}a_{n+k-1} + y_{k-2}a_{n+k-2} + \dots + y_{1}a_{n+1} + y_{0}a_{n} \qquad n \geq 0 \quad a_0 = 1, a_1 = 4$$
\item \textbf{Beispiel:} $$a_{n+2} = 3a_{n+1} - 2a_n \qquad n \geq 0 \quad a_0 = 1, a_1 = 4$$
\end{itemize}
\item \textbf{Inhomogene lineare Rekursionsgleichungen mit konstanten Koeffizienten:}
\begin{itemize}
\item Hat zusätzliche inhomogene Komponente, entweder eine Funktion von $n$ ($n$, $n^2$, $n^3$, $\dots$, $x^n$, $\dots$) oder eine konstante Zahl
\item \textbf{Aufbau:} $$a_{n+k} = y_{k-1}a_{n+k-1} + y_{k-2}a_{n+k-2} + \dots + y_{1}a_{n+1} + y_{0}a_{n} + f(n) \qquad n \geq 0 \quad a_0 = 1, a_1 = 4$$
\item \textbf{Beispiel:} $$a_{n+2} = 3a_{n+1} - 2a_n + 4n \qquad n \geq 0 \quad a_0 = 1, a_1 = 4$$ \begin{center}\textit{oder}
\end{center} $$a_{n+2} = 3a_{n+1} - 2a_n + 6 \qquad n \geq 0 \quad a_0 = 1, a_1 = 4$$ \begin{center}\textit{oder}
\end{center} $$a_{n+2} = 3a_{n+1} - 2a_n + 3^n \qquad n \geq 0 \quad a_0 = 1, a_1 = 4$$ \begin{center}\textit{oder}  $$a_{n+2} = 3a_{n+1} - 2a_n + 4n + 6 + 3^n \qquad n \geq 0 \quad a_0 = 1, a_1 = 4$$
\end{center}
\end{itemize}
\end{itemize}\

\textbf{2. Grad der Rekursionsgleichung bestimmen}

\begin{itemize}
\item Der Grad einer Rekursionsgleichung bezieht sich auf den höchsten Exponenten der Rekursionsvariable
\item \textbf{Beispiel:}$a_{n+{\color{red}2}} = 3a_{n+1} - 2a_n + 6 \rightarrow$ Grad $= {\color{red}2}$
\end{itemize}






