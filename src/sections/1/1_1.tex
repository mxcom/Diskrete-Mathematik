\subsection{Anzahl Aufteilungen von einer Menge $N$ von Kugeln in eine Menge $R$ von Fächern}

\begin{table}[h]
\centering
\begin{tabular}{l|c|l|l|l}
$|N| = n, |R| = r$ & beliebig & injektiv & surjektiv & bijektiv\\
\hline
% === %
$N$ unterscheidbar & & $r \geq n : r^{\underline{n}}$ & $n \geq r : r!S_{n,r}$ & $r = n : n!$\\
$R$ unterscheidbar & $r^n$ & $r < n : 0$ & $n < r : 0$ & $r \neq n : 0$\\
\hline
% === %
$N$ nicht unterscheidbar & $\begin{pmatrix}r + n - 1\\ n\end{pmatrix}$ & $r \geq n : \begin{pmatrix}r\\ n\end{pmatrix}$ & $n \geq r: \begin{pmatrix}n-1\\ r-1\end{pmatrix}$ & $r = n : 1$\\
$R$ unterscheidbar & $\displaystyle = \frac{r^{\overline{n}}}{n!}$ & $r < n : 0$ & $n < r : 0$ & $r \neq n : 0$\\
\hline
% === %
$N$ unterscheidbar & & $r \geq n : 1$ &  $n \geq r : S_{n,r}$ & $r = n : 1$\\
$R$ nicht unterscheidbar & $\displaystyle \sum_{k=1}^{r}S_{n,k}$ & $r < n : 0$ & $n < r : 0$ & $r \neq n : 0$\\
\hline
% === %
$N$ nicht unterscheidbar & & $r \geq n : 1$ & $n \geq r : P_{n,r}$ & $r = n : 1$\\
$R$ nicht unterscheidbar & $\displaystyle \sum_{k=1}^{r} P_{n,k}$ & $r < n : 0$ & $n < r : 0$ & $r \neq n : 0$
\end{tabular}
\end{table}

\textbf{Beliebige Aufteilung:}

\begin{itemize}[leftmargin=*]
\item Keine speziellen Regeln bei der Aufteilung (Man kann die Kugeln verteilen wie man möchte)
\begin{itemize}
\item[$\Rightarrow$] Ein Fach darf leer sein, kann eine Kugel haben oder mehrere Kugeln haben
\end{itemize}
\item Alle Kugeln müssen benutzt werden
\end{itemize}\

\textbf{Injektive Aufteilung:}


\begin{itemize}[leftmargin=*]
\item Jedes Fach darf höchstens eine Kugel enthalten (Kein Fach darf mehr als eine Kugel enthalten)
\begin{itemize}
\item[$\Rightarrow$] Fächer dürfen leer bleiben (wenn Anzahl Fächer $>$ Anzahl Kugeln)
\end{itemize}
\item Nicht jede Kugel muss benutzt werden
\end{itemize}\

\textbf{Surjektive Aufteilung:}

\begin{itemize}[leftmargin=*]
\item Jedes Fach muss mindestens eine Kugel enthalten
\begin{itemize}
\item[$\Rightarrow$] Kein Fach darf leer sein
\item[$\Rightarrow$] Einige Fächer können mehr Kugeln haben als andere (wenn Anzahl Fächer $<$ Anzahl Kugeln)
\end{itemize}
\item Nicht jede Kugel muss benutzt werden
\end{itemize}\

\textbf{Bijektive Aufteilung:}

\begin{itemize}[leftmargin=*]
\item Jedes Fach muss genau eine Kugel enthalten
\begin{itemize}
\item[$\Rightarrow$] Kein Fach darf leer sein
\item[$\Rightarrow$] Kein Fach darf mehre Kugeln enthalten
\item[$\Rightarrow$] Anzahl Kugeln = Anzahl Fächer
\end{itemize}
\item Jede Kugel muss somit benutzt werden
\end{itemize}\

