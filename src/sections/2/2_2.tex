\subsection{Codewörter sind gegeben}

\subsubsection*{$n$ und $m$ bestimmen}

\begin{table}[h]
\centering
\begin{tabular}{c}
$n$ = Länge der Codewörter $\Rightarrow n = 5$\\
{\color{blue}00000}\\
01101\\
10111\\
11010\\
\end{tabular}
\end{table}

% === %

\begin{table}[h]
\centering
\begin{tabular}{c}
$m$ = Dimension $\Rightarrow m = 2$\\
{\color{blue}00}000\\
{\color{blue}01}101\\
{\color{blue}10}111\\
{\color{blue}11}010\\
\end{tabular}
\end{table}

\subsubsection*{Hamming-Distanz/Code-Distanz/$d(c)$}

\begin{itemize}
\item Hemming-Distanz ist die minimale Änderung des Gewichts
\item Wird mit $d(C)$ bezeichnet
\end{itemize}\

Beispiel:

\begin{itemize}
\item Anzahl Einsen in Codewörter Zählen
\item $0 \dots 0$ wird dabei nicht beachtet
\end{itemize}

\begin{table}[h]
\begin{tabular}{ccc}
01101 & $\rightarrow$ & Gewicht: 3\\
10111 & $\rightarrow$ & Gewicht: 4\\
11010 & $\rightarrow$ & Gewicht: 3\\
\end{tabular}
\end{table}

\begin{itemize}
\item Gewicht der Codewörter vergleichen und minimalstes auswählen
\item[$\Rightarrow$] $d(C) = 3$, da es minimal ist
\end{itemize}

\subsubsection*{t-fehlererkennend}

\subsubsection*{t-fehlerkorrigierend}

\subsubsection*{t-ausfällekorrigierend}

\newpage

\subsubsection*{Kanonische Generatormatrix}

\begin{itemize}
\item Definition: Aus der Generatormatrix kann man alle möglichen Codewörter der Sprache erzeugen.
\item Falls die Generatormatrix gegeben ist, lässt sich die die kanonische Generatormatrix durch das Anwenden des Gaußschen Verfahrens auf die Generatormatrix erstellen.
\item Größe: $(m \times n)$
\item Aufbau: $G = \begin{pmatrix} E & G' \end{pmatrix}$
\begin{itemize}
\item ${\color{red}E}$: Einheitsmatrix
\item ${\color{ForestGreen}G'}$: Linear unabhängiger Rest aus Codewörtern
\end{itemize}

\end{itemize}

$$
G = \begin{pmatrix}
{\color{red}1} & {\color{red}0} & {\color{ForestGreen}1} & {\color{ForestGreen}1} & {\color{ForestGreen}1}\\
{\color{red}0} & {\color{red}1} & {\color{ForestGreen}1} & {\color{ForestGreen}0} & {\color{ForestGreen}1}
\end{pmatrix}
$$

\subsubsection*{Kanonische Kontrollmatrix}

\begin{itemize}
\item Definition: Erfüllt ein Wort $wort \cdot kontrollmatrix = 0$ ist es ein richtiges Codewort.
\item Größe: $n \times (n - m)$
\item Aufbau: $H = \begin{pmatrix}-G\\E\end{pmatrix}$
\begin{itemize}
\item ${\color{red}E}$: Einheitsmatrix
\item ${\color{ForestGreen}-G}$: Linear unabhängiger Rest aus Codewörtern
\end{itemize}
\end{itemize}

$$
H = \begin{pmatrix}
{\color{ForestGreen}-1} & {\color{ForestGreen}-1} & {\color{ForestGreen}-1}\\
{\color{ForestGreen}-1} & {\color{ForestGreen}-0} & {\color{ForestGreen}-1}\\
{\color{red}1} & {\color{red}0} & {\color{red}0}\\
{\color{red}0} & {\color{red}1} & {\color{red}0}\\
{\color{red}0} & {\color{red}0} & {\color{red}1}
\end{pmatrix} = \begin{pmatrix}
{\color{ForestGreen}1} & {\color{ForestGreen}1} & {\color{ForestGreen}1}\\
{\color{ForestGreen}1} & {\color{ForestGreen}0} & {\color{ForestGreen}1}\\
{\color{red}1} & {\color{red}0} & {\color{red}0}\\
{\color{red}0} & {\color{red}1} & {\color{red}0}\\
{\color{red}0} & {\color{red}0} & {\color{red}1}
\end{pmatrix}
$$

Modulo mit negativen Zahlen:

\begin{table}[h]
\begin{tabular}{c|cccccccccccccccccccc}
& $-10$ & $-9$ & $-8$ & $-7$ & $-6$ & $-5$ & $-4$ & $-3$ & $-2$ & $-1$ & $0$ & $1$ & $2$ & $3$ & $4$ & $5$ & $6$ & $7$ & $8$ & $9$\\
\hline
$\mathbb{Z}_2$ & & & & & & & & & $0$ & $1$ & $[0$ & $1]$ & & & & & & & &\\
\hline
$\mathbb{Z}_3$ & & & & & & & & $0$ & $1$ & $2$ & $[0$ & $1$ & $2]$ & & & & & & &\\
\hline
$\mathbb{Z}_4$ & & & & & & & $0$ & $1$ & $2$ & $3$ & $[0$ & $1$ & $2$ & $3]$ & & & & & &\\
\hline
$\mathbb{Z}_5$ & & & & & & $0$ & $1$ & $2$ & $3$ & $4$ & $[0$ & $1$ & $2$ & $3$ & $4]$ & & & & &\\
\hline
$\mathbb{Z}_6$ & & & & & $0$ & $1$ & $2$ & $3$ & $4$ & $5$ & $[0$ & $1$ & $2$ & $3$ & $4$ & $5]$ & & & &\\
\hline
$\mathbb{Z}_7$ & & & & $0$ & $1$ & $2$ & $3$ & $4$ & $5$ & $6$ & $[0$ & $1$ & $2$ & $3$ & $4$ & $5$ & $6]$ & & &\\
\hline
$\mathbb{Z}_8$ & & & $0$ & $1$ & $2$ & $3$ & $4$ & $5$ & $6$ & $7$ & $[0$ & $1$ & $2$ & $3$ & $4$ & $5$ & $6$ & $7]$ & &\\
\hline
$\mathbb{Z}_9$ & & $0$ & $1$ & $2$ & $3$ & $4$ & $5$ & $6$ & $7$ & $8$ & $[0$ & $1$ & $2$ & $3$ & $4$ & $5$ & $6$ & $7$ & $8]$ &\\
\hline
$\mathbb{Z}_{10}$ & $0$ & $1$ & $2$ & $3$ & $4$ & $5$ & $6$ & $7$ & $8$ & $9$ & $[0$ & $1$ & $2$ & $3$ & $4$ & $5$ & $6$ & $7$ & $8$ & $9]$\\
\end{tabular}
\end{table}