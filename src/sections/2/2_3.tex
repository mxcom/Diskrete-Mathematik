\subsection{Syndromtabelle}

\begin{itemize}
\item Definition: Identifiziert und korrigiert Fehler in Codewörter durch Vergleich mit erwarteten Werten.
\item Anzahl Zeilen: $q^{n-m}$
\item Anzahl Spalten: $n$
\end{itemize}

Beispiel: $(7 \times 3)$-Kontrollmatrix, $n = 7, m = 4$

\begin{minipage}{0.2\textwidth}

$$
  H = \begin{pmatrix}
  {\color{red}1} & {\color{red}1} & {\color{red}1}\\
  {\color{blue}1} & {\color{blue}0} & {\color{blue}1}\\
  {\color{Green}0} & {\color{Green}1} & {\color{Green}1}\\
  1 & 1 & 0\\
  1 & 0 & 0\\
  0 & 1 & 0\\
  0 & 0 & 1
  \end{pmatrix}
$$
\end{minipage}
\hfill
\begin{minipage}{0.2\textwidth}

\begin{tabular}{c|c}
    $a$ & $S = a \cdot H$\\
    \hline
    0000000 & 000\\
    1000000 & {\color{red}111}\\
    0100000 & {\color{blue}101}\\
    0010000 & {\color{Green}011}\\
    0001000 & 110\\
    0000100 & 100\\
    0000010 & 010\\
    0000001 & 001\\
\end{tabular}
\end{minipage}
\hfill
\begin{minipage}{0.5\textwidth}

\begin{itemize}[leftmargin=*]
\item Anzahl Zeilen: $2^{7-4} = 8$
\item Anzahl Spalten: $7$
\item Wenn über $0000001$ hinaus geht egal ob $1000001$, $1100000$, \dots
\end{itemize}
\end{minipage}\

1. Überprüfen ob es ein empfangenes Codewort fehlerfrei ist$(wort \cdot kontrollmatrix = 0)$

\begin{itemize}
\item Empfangenes Wort: $y = 1010010$
\end{itemize}\

$1010010 \cdot \begin{pmatrix}
1 & 1 & 1\\
1 & 0 & 1\\
0 & 1 & 1\\
1 & 1 & 0\\
1 & 0 & 0\\
0 & 1 & 0\\
0 & 0 & 1
\end{pmatrix} = 110 \not = 0 \Rightarrow$ Fehler im empfangenen Codewort\\

2. Codewort durch Syndromtabelle korrigieren

\begin{itemize}
\item Klassenanführer von $110$ aus Syndromtabelle ablesen: $a =0001000$
\item $empfangenes \ Codewort - Klassenanfuehrer = korrigiertes \ Codewort$
\end{itemize}

\begin{table}[h]
\centering
\begin{tabular}{ccc}
$y$ & & $1010010$\\
$a$ & $-$ & $0001000$\\
\hline
& & $1011010$
\end{tabular}
\end{table}

\begin{itemize}
\item korrigiertes Codewort: $1011010$
\end{itemize}\

3. Nachricht extrahieren

\begin{itemize}
\item Letzte Stellen des korrigierten Codewort entfernen, um die Nachricht zu erhalten (Anzahl entfernte Stellen entspricht Länge des Syndrom).
\item Nachricht: 1011\sout{010} = 1011
\end{itemize}