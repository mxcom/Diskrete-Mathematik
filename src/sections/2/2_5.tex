\subsection{Kontrollmatrix zu Generatormatrix}

\begin{itemize}
\item Die Kontrollmatrix eines linearen (n,m)-Codes C über dem Körper $\mathbb{Z}_5$
\item Da $\mathbb{Z}_5$ alles $mod \ 5$
\end{itemize}

$$
H = \begin{pmatrix}
4 & 1\\
1 & 1\\
2 & 4\\
3 & 4
\end{pmatrix} = \begin{pmatrix}
4 & 1 & 2 & 3\\
1 & 1 & 4 & 4
\end{pmatrix}^T
$$

\begin{itemize}
\item Um aus einer Kontrollmatrix eine Generatormatrix abzuleiten machen wir uns die Eigenschaft $G\cdot H = 0$ zu nutze
\item Um $G\cdot H = 0$ zu lösen stellen wir eine Tabelle auf die unser Gleichungssystem repräsentiert
\end{itemize}

\begin{table}[h]
\centering
\begin{tabular}{cccc|c}
$g_1$ & $g_2$ & $g_3$ & $g_4$ &\\
\hline
4 & 1 & 2 & 3 & 0\\
1 & 1 & 4 & 4 & 0
\end{tabular}
\end{table}

\begin{itemize}
\item Nun wählt man 2 beliegt Variablen (Anzahl Variablen: $|g| - 2$)
\item $g_3 = \lambda$, $g_4 = \mu$
\item Nun löst man das Gleichungssystem für die restlichen $g$'s ($g_1$ und $g_2$)
\item \textbf{Achtung:} Wir rechnen in $\mathbb{Z}_5$
\begin{itemize}
\item Wenn man neative Zahlen hat gilt $-a \ mod \ 5 = b$ und man rechnet mit der positive Zahl weiter
\item Wenn man dividiert immer das multiplikative Inverse der Zahl (z.B $a$) nehmen, und mit dem Inversen (z.B $b$) multiplizieren ($a \cdot b \ mod  \ 5 = 1$)
\end{itemize}
\end{itemize}

\textbf{Nach $g_1$ lösen:}
\begin{align*}
0 &= 4g_1 + g_2 + 2g_3 + 3g_4\\
4g_1 &= -g_2 -2g_3 -3g_4\\
4g_1 &= 4g_2 + 3g_3 + 2g_4 \ | \ variablen \ einsetzen\\
4g_1 &= 4g_2 + 3\lambda + 2\mu \ | \cdot 4 \ (mult. \ Inverses \ von \ 4)\\
\end{align*}
$$
\boxed{g_1 = g_2 + 2\lambda + 3\mu} \xrightarrow[]{\text{Nach dem lösen von } g_2} \boxed{g_1 = 4\lambda + 2\mu}
$$

\textbf{Nach $g_2$ lösen:}
\begin{align*}
0 &= g_1 + g_2 + 4g_3 + 4g_4\\
g_2 &= -g_1 - 4g_4 - 4g_4 \ | g_1 \ und \ variablen \ einsetzen\\
&= -(g_2 + 2\lambda + 3\mu) - 4\lambda - 4\mu\\
&= -g_2 + -2\lambda -3\mu - 4\lambda - 4\mu\\
&= 4g_2 + 3 \lambda + 2\mu + 1 \lambda + 1\mu\\
&= 4g_2 + 4 \lambda + 3 \mu \ | -4g_2\\
2g_2 &= 4 \lambda + 3 \mu \ | \cdot 3 \ (mult. \ Inverses \ von \ 2)
\end{align*}
$$
\boxed{g_2 = 2\lambda + 4\mu}
$$

\begin{itemize}
\item Nun lässt sich $G$ bestimmen
\item $G$ hat die Form $(g_1, g_2, g_3, g_4)$
\end{itemize}
$$
G = (4\lambda + 2\mu, 2\lambda + 4\mu, \lambda, \mu)
$$

\begin{itemize}
\item Nun muss mann für alle Spalten $\lambda$ und $\mu$ so bestimmen das $G\cdot H = 0$ aufgeht
\item \textbf{Für die ersten Spalte:} $\lambda = 1, \mu = 0$
\item \textbf{Für die zweite Spalte:} $\lambda = 0, \mu = 1$
\end{itemize}
$$
G = \begin{pmatrix}
4 & 2 & 1 & 0\\
2 & 4 & 0 & 1
\end{pmatrix} = \begin{pmatrix}
4 & 2\\
2 & 4\\
1 & 0\\
0 & 1
\end{pmatrix}^T
$$